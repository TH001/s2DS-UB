\documentclass[12pt,a4paper]{scrartcl}
\usepackage[ngerman]{babel}
\usepackage[utf8]{inputenc}
\usepackage{amsmath, amsfonts, amssymb}
\usepackage{lastpage}
\usepackage{fancyhdr}
\usepackage{listings}
\usepackage{hyperref}
\usepackage{graphicx}
\usepackage{tikz}
\usepackage{chemfig}
\usepackage{chemformula}
\usepackage{ngerman}
\usepackage{color}
\usepackage{arydshln}
\usepackage{verbatim}
\usepackage{ulem}
\usepackage{amsthm}
\usepackage{enumerate}
\usepackage{geometry}
\usepackage{lscape}
\usepackage{stmaryrd}
\usepackage{array}
\usepackage{longtable}
\newcommand{\RM}[1]{\MakeUppercase{\romannumeral #1{}}}
\newcommand{\leadingzero}[1]{\ifnum #1<10 0\the#1\else\the#1\fi}
\newcommand{\todayshort}{\leadingzero{\day}.\leadingzero{\month}.\the\year} 
\newcommand{\equ}{\Leftrightarrow} 
\renewcommand{\headrulewidth}{0.4pt}
\renewcommand{\footrulewidth}{0.4pt}
\usetikzlibrary{automata}
\setcounter{section}{-1}

\usepackage{tabularx}
\newcolumntype{L}[1]{>{\raggedright\arraybackslash}p{#1}} % linksbündig mit Breitenangabe
\newcolumntype{C}[1]{>{\centering\arraybackslash}p{#1}} % zentriert mit Breitenangabe
\newcolumntype{R}[1]{>{\raggedleft\arraybackslash}p{#1}} % rechtsbündig mit Breitenangabe

\definecolor{white}{rgb}{1,1,1}
\definecolor{darkred}{rgb}{0.3,0,0}
\definecolor{darkgreen}{rgb}{0,0.3,0}
\definecolor{darkblue}{rgb}{0,0,0.3}
\definecolor{pink}{rgb}{0.78,0.09,0.51}
\definecolor{purple}{rgb}{0.28,0.24,0.55}
\definecolor{orange}{rgb}{1,0.6,0.0}
\definecolor{grey}{rgb}{0.4,0.4,0.4}
\definecolor{lightgrey}{rgb}{0.9,0.9,0.9}
\definecolor{aquamarine}{rgb}{0.4,0.8,0.65}
\definecolor{red}{rgb}{0.75,0,0}
\definecolor{green}{rgb}{0,0.6,0}
\definecolor{blue}{rgb}{0,0,0.6}
\definecolor{yello}{rgb}{0,0.16,0.16}

\definecolor{commentgreen}{rgb}{0,0.6,0}
\definecolor{mauve}{rgb}{0.58,0,0.82}
\definecolor{javadocblue}{rgb}{0.25,0.35,0.75}

\lstset{ %
	backgroundcolor=\color{white},
	basicstyle=\footnotesize,
	%breaklines=true,
	%captionpos=b,
	commentstyle=\color{commentgreen},
	escapeinside={\%*}{*)},
	keywordstyle=\color{blue},
	stringstyle=\color{mauve},
	morecomment=[s][\color{javadocblue}]{/**}{*/},
	frame=single,
	tabsize=4,
}

\author{Tom Heine}

\title{Abgabe: algodat-blatt01}

\date{\today}

\geometry{a4paper,
	left=25mm,
	right=25mm,
	top=20mm%,bottom=30mm
}
\pagestyle{fancy}
\fancyhead[L]{
	\begin{small}
		Digitale Systeme\\
		SOSE 2019
	\end{small}
}
% z.B Name und Matrikelnummer

% Zentraler Teil des Headers
\fancyhead[C]{
	\begin{large}
		Abgabe: Blatt01
	\end{large}\\
	Version 04.06.2019
} 
% z.B. Name der Veranstaltung

% Rechter Teil des Headers
\fancyhead[R]{
	\begin{small}
		%		\begin{tabular}{|rr|}\hline
		Abgabegruppe: AG42%\\\hline
		%\multicolumn{1}{r}{Datum:}&\multicolumn{1}{r}{\todayshort}
		%		\end{tabular}
	\end{small}
} 
\fancyfoot[C]{
	\thepage\space von \pageref{LastPage}
}
\headheight=25pt
\begin{document}
	%	\maketitle
	%	\tableofcontents
	\newpage
	\begin{center}
		\begin{tabular}{ccc}
			Gerlach, Luisa&gerlaclu&599244\\
			Heine, Tom Martin&heinetom&597978\\
			Kühne, Marc Sebastian&kuehnese&599833\\
			Seegert, Noah-Joël&segertno&596234
		\end{tabular}
	\end{center}
	\section*{Aufgabe 1}
	\subsection*{(a) 0 001 000010 000100}
	MOV R2, R4
	\subsection*{(b) 0 110 010111 000011 |
		0 000 000000 010001}
	ADD R23, R3, 17
	\subsection*{(c) 0 000 110101 010010}
	PUSH R18
	\newpage
	\section*{Aufgabe 2}
	
	Aufbauen eines Befehlsformats: \\
	\\
	\begin{tabular}{c l}
		Speicher Wort- / Byteadressiert &$=>$ 1 Bit\\
		16 Befehle &$=>$ 4 Bit \\
		16 Register &$=>$ 4 Bit \\
		$16*1024*1024$ Speicheradr. &$=>$ 24 Bit \\
		5 Adressarten für 1. Operanten &$=>$ 3 Bit \\
		1 Adressarten für 2. Operanten &$=>$ immer gleich (kein Bit) \\
		meisten Platz für 1.Operanten - speicherdirekt &$=>$ 24 Bit \\
		meisten Platz für 2.Operanten - registerdirekt &$=>$ 4 Bit \\
	\end{tabular} \\
	\\
	Zusammenfügen zu einem Befehl einer 2-Adressmaschine: \\
	\begin{tabular}{c | c | c | c | c}
		Wort/Byte & Opcode & Adr.Art 1 & Operant 1 & Operant 2 \\
		1 Bit & 4 Bit & 3 Bit & 24 Bit & 4 Bit
	\end{tabular} \\
	\\
	Gesamt 36 Bit. \\
	Da aber ganze Vielfache von Worten(16 Bit) gefordert, muss die Befehlslänge mit 12  'Reservebits' aufgefüllt werden, sodass sich folgendes ergibt: \\
	\begin{tabular}{c | c | c | c | c | c}
		r & b & x & i & a & b \\
		Reservebits & Wort/Byte & Opcode & Adr.Art 1 & Operant 1 & Operant 2 \\
		12 Bit & 1 Bit & 4 Bit & 3 Bit & 24 Bit & 4 Bit
	\end{tabular} \\
	\\
	Befehlsformat: rrrrrrrrrrrr b xxxx iii aaaaaaaaaaaaaaaaaaaaaaaa bbbb \\
	48 Bit Befehle 
	
	\newpage
	%\begin{landscape}
	
	\section*{Aufgabe 3}
	
	\begin{comment}
	\begin{center}
	\begin{tabular}{|c |c | c|} 
	\hline
	Adresse & Mikrobefehl & Kommentar\\
	\hline\hline
	0 & RD & ließt Befehl ein\\ 
	\hline
	1 & WRDY& wartet, bis Speicherzugriff beendet wurde \\
	\hline
	2 & DECODE & dekodiert IR\\
	\hline
	3 & IR\_op\_out& ließt IR\\
	\hline
	4 & SR = Rop2 & wählt Register 2\\
	\hline
	5 & R\_out & schreibt in Register 2\\
	\hline
	6 & WRDY& wartet, bis Speicherzugriff beendet wurde \\
	\hline
	\end{tabular}
	\end{center}
	\end{comment}
	
	\begin{table}[h]
		\begin{longtable}{|l|l|L{5.7cm}|} 
			\hline
			%\toprule
			Adresse & Mikrobefehl & Kommentar\\
			\hline\hline
			%\midrule\midrule\endhead
			0 & IP\_out, MAR\_in, RD, A\_in, F = A+1, NA[1] & Befehl wird aus dem Hauptspeicher in die ALU geladen\\ \hline
			1 & ALU\_out, IP\_in, WRDY, NA[2] & wartet, bis Speicherzugriff beendet wurde \\ \hline
			2 & MDR\_out, IR\_in, DECODE[jpz: 3, sub: 8] & Befehl wird gelesen und dekodiert\\
			\hline\hline
			3 & IP\_out, MAR\_in, RD,  A\_in, F = A+1, NA[4] & Sprungdistanz wird aus dem Hauptspeicher in die ALU gelesen\\ \hline
			4 & ALU\_out, IP\_in, WRDY, NA[5] & wartet, bis Speicherzugriff beendet wurde \\ \hline
			5 & IFZ[6] & prüft, ob die Flag 0 ist\\ \hline
			6 & MDR\_out, B\_in, F=A+B, NA[7] & addiert Sprungdistanz in ALU zu IP\\ \hline
			7 & ALU\_out, IP\_in, NA[0] & schreibt IP zurück \\
			\hline\hline
			8 & SR=ROP1, R\_out, MAR\_in, RD, NA[9] & erster Operand wird adressiert\\ \hline
			9 & SR=ROP2, R\_out, A\_in, NA[10] & zweiter Operand wird in ALU geladen\\ \hline
			10 & IRop\_out, B\_in, F=A+B, WRDY, NA[11] & Adresse des zweiten Operanden wird berechnet, wartet bis Speicherzugriff beendet\\ \hline
			11& MDR\_out, Y\_in, NA[12]& erster Operand wird gespeichert\\ \hline
			12&ALU\_out, MAR\_in, RD, NA[13]& zweiter Operandwird adressiert\\ \hline
			13& Y\_out, B\_in, WRDY, NA[14] & ertser Operand wird in die ALU geladen\\ \hline
			14& MDR\_out, A\_in, F=B-A, FLE, NA[15]& zweiter Operand wird ALU geladen und Subtraktion wird ausgeführt\\ \hline
			15& ALU\_out, MDR\_in, WR, WRDY,  NA[0]& Ergebnis wird in den Speicher geschrieben und es wird gewartet, bis der Speicherzugriff beendet wurde\\
			\hline
			%\bottomrule
		\end{longtable}
	\end{table}
	%\end{landscape}
\end{document}
