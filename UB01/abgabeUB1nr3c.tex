\documentclass[12pt,a4paper]{scrartcl}
\usepackage[ngerman]{babel}
\usepackage[utf8]{inputenc}
\usepackage{amsmath, amsfonts, amssymb}
\usepackage{lastpage}
\usepackage{fancyhdr}
\usepackage{listings}
\usepackage{hyperref}
\usepackage{graphicx}
\usepackage{tikz}
\usepackage{chemfig}
\usepackage{chemformula}
\usepackage{ngerman}
\usepackage{color}
\usepackage{arydshln}
\usepackage{verbatim}
\usepackage{ulem}
\usepackage{amsthm}
\usepackage{enumerate}
\usepackage{geometry}
\usepackage{stmaryrd}
\newcommand{\RM}[1]{\MakeUppercase{\romannumeral #1{}}}
\newcommand{\leadingzero}[1]{\ifnum #1<10 0\the#1\else\the#1\fi}
\newcommand{\todayshort}{\leadingzero{\day}.\leadingzero{\month}.\the\year} 
\newcommand{\equ}{\Leftrightarrow} 
\renewcommand{\headrulewidth}{0.4pt}
\renewcommand{\footrulewidth}{0.4pt}
\usetikzlibrary{automata}
\setcounter{section}{-1}

\definecolor{white}{rgb}{1,1,1}
\definecolor{darkred}{rgb}{0.3,0,0}
\definecolor{darkgreen}{rgb}{0,0.3,0}
\definecolor{darkblue}{rgb}{0,0,0.3}
\definecolor{pink}{rgb}{0.78,0.09,0.51}
\definecolor{purple}{rgb}{0.28,0.24,0.55}
\definecolor{orange}{rgb}{1,0.6,0.0}
\definecolor{grey}{rgb}{0.4,0.4,0.4}
\definecolor{aquamarine}{rgb}{0.4,0.8,0.65}
\definecolor{red}{rgb}{0.75,0,0}
\definecolor{green}{rgb}{0,0.6,0}
\definecolor{blue}{rgb}{0,0,0.6}
\definecolor{yello}{rgb}{0,0.16,0.16}

\author{Tom Heine}

\title{Abgabe:}

\date{\today}

\geometry{a4paper,
	left=25mm,
	right=25mm,
	top=20mm%,bottom=30mm
}

\pagestyle{fancy}
\fancyhead[L]{
	\begin{tabular}{l}
		Heine, Tom Martin\\
		Kühne, Marc Sebastian\\
	\end{tabular}
}	
% z.B Name und Matrikelnummer
% Zentraler Teil des Headers
\fancyhead[C]{
	\begin{large}
			Abgabe: 
	\end{large}\\
	Version vom 16.10.2018
} 
% z.B. Name der Veranstaltung

% Rechter Teil des Headers 
\fancyhead[R]{
	\begin{tabular}{|rr|}\hline
		Abgabegruppe:& \\
		Wochentag:& \\\hline
		\\
	\end{tabular}
} 

\fancyfoot[C]{
	\thepage\space von \pageref{LastPage}
}
\fancyfoot[R]{
	Datum: \todayshort
}

\headheight=45pt
\begin{document}
%\maketitle
%\tableofcontents
\newpage
	
\section*{Aufgabe 3c}
	$s = a + b$, wobei sich das Ergebnis s für die Addition von zwei 4-Bit Zahlen a,b sich wie gefolgt in Bitschreibweise zusammensetzt: $s = c_4 s_3s_2s_1s_0$ \\
	$s_i = a_i \oplus b_i \oplus c_i$ für $i \in \{0,1,2,3\}$ mit $c_0$ als $c_{in}$ und $c_4$ als letzten Übertrag \\
	\\
	$s_0 = a_0 \oplus b_0 \oplus c_0$ \\
	$s_1 = a_1 \oplus b_1 \oplus c_1$ \\
	$s_2 = a_2 \oplus b_2 \oplus c_2$ \\
	$s_3 = a_3 \oplus b_3 \oplus c_3$ \\
	$c_4$ \\
	\\
	\\
	Die Carries $c_i$ lassen mit den folgenden Schaltfunktionen vorberechnen \\
	$c_i = g_{i-1} \lor p_{i-1}c_{i-1}$ mit $g_i = a_ib_i$ und $p_i = a_i \lor b_i$, sodass \\
	\\
	$c_1 = g_0 \lor p_0c_0$  (Einmal ausführlich: $c_1 = a_0b_0 \lor (a_0 \lor b_0)c_0$)\\
	$c_2 = g_1 \lor p_1g_0 \lor p_1p_0c_0$ \\
	$c_3 = g_2 \lor p_2g_1 \lor p_2p_1g_0 \lor p_2p_1p_0c_0$ \\
	$c_4 = g_3 \lor p_3g_2 \lor p_3p_2g_1 \lor p_3p_2p_1g_0 \lor p_3p_2p_1p_0c_0$ \\
\end{document}
