\documentclass[12pt,a4paper]{scrartcl}
\usepackage[ngerman]{babel}
\usepackage[utf8]{inputenc}
\usepackage{amsmath, amsfonts, amssymb}
\usepackage{lastpage}
\usepackage{fancyhdr}
\usepackage{listings}
\usepackage{hyperref}
\usepackage{graphicx}
\usepackage{tikz}
\usepackage{chemfig}
\usepackage{chemformula}
\usepackage{ngerman}
\usepackage{color}
\usepackage{arydshln}
\usepackage{verbatim}
\usepackage{ulem}
\usepackage{amsthm}
\usepackage{enumerate}
\usepackage{geometry}
\usepackage{lscape}
\usepackage{stmaryrd}
\usepackage{array}
\usepackage{longtable}
\newcommand{\RM}[1]{\MakeUppercase{\romannumeral #1{}}}
\newcommand{\leadingzero}[1]{\ifnum #1<10 0\the#1\else\the#1\fi}
\newcommand{\todayshort}{\leadingzero{\day}.\leadingzero{\month}.\the\year} 
\newcommand{\equ}{\Leftrightarrow} 
\renewcommand{\headrulewidth}{0.4pt}
\renewcommand{\footrulewidth}{0.4pt}
\usetikzlibrary{automata}
\setcounter{section}{-1}

\usepackage{tabularx}
\newcolumntype{L}[1]{>{\raggedright\arraybackslash}p{#1}} % linksbündig mit Breitenangabe
\newcolumntype{C}[1]{>{\centering\arraybackslash}p{#1}} % zentriert mit Breitenangabe
\newcolumntype{R}[1]{>{\raggedleft\arraybackslash}p{#1}} % rechtsbündig mit Breitenangabe

\definecolor{white}{rgb}{1,1,1}
\definecolor{darkred}{rgb}{0.3,0,0}
\definecolor{darkgreen}{rgb}{0,0.3,0}
\definecolor{darkblue}{rgb}{0,0,0.3}
\definecolor{pink}{rgb}{0.78,0.09,0.51}
\definecolor{purple}{rgb}{0.28,0.24,0.55}
\definecolor{orange}{rgb}{1,0.6,0.0}
\definecolor{grey}{rgb}{0.4,0.4,0.4}
\definecolor{lightgrey}{rgb}{0.9,0.9,0.9}
\definecolor{aquamarine}{rgb}{0.4,0.8,0.65}
\definecolor{red}{rgb}{0.75,0,0}
\definecolor{green}{rgb}{0,0.6,0}
\definecolor{blue}{rgb}{0,0,0.6}
\definecolor{yello}{rgb}{0,0.16,0.16}

\definecolor{commentgreen}{rgb}{0,0.6,0}
\definecolor{mauve}{rgb}{0.58,0,0.82}
\definecolor{javadocblue}{rgb}{0.25,0.35,0.75}

\lstset{ %
	backgroundcolor=\color{white},
	basicstyle=\footnotesize,
	%breaklines=true,
	%captionpos=b,
	commentstyle=\color{commentgreen},
	escapeinside={\%*}{*)},
	keywordstyle=\color{blue},
	stringstyle=\color{mauve},
	morecomment=[s][\color{javadocblue}]{/**}{*/},
	frame=single,
	tabsize=4,
}

\author{Tom Heine}

\title{Abgabe: algodat-blatt03}

\date{\today}

\geometry{a4paper,
	left=25mm,
	right=25mm,
	top=20mm%,bottom=30mm
}
\pagestyle{fancy}
\fancyhead[L]{
	\begin{small}
		Digitale Systeme\\
		SOSE 2019
	\end{small}
}
% z.B Name und Matrikelnummer

% Zentraler Teil des Headers
\fancyhead[C]{
	\begin{large}
		Abgabe: Blatt03
	\end{large}\\
	Version 04.06.2019
} 
% z.B. Name der Veranstaltung

% Rechter Teil des Headers
\fancyhead[R]{
	\begin{small}
%		\begin{tabular}{|rr|}\hline
			Abgabegruppe: AG42%\\\hline
			%\multicolumn{1}{r}{Datum:}&\multicolumn{1}{r}{\todayshort}
%		\end{tabular}
	\end{small}
} 
\fancyfoot[C]{
	\thepage\space von \pageref{LastPage}
}
\headheight=25pt
\begin{document}
	%	\maketitle
	%	\tableofcontents
	\newpage
	\begin{center}
		\begin{tabular}{ccc}
			Gerlach, Luisa&gerlaclu&599244\\
			Heine, Tom Martin&heinetom&597978\\
			Kühne, Marc Sebastian&kuehnese&599833\\
			Seegert, Noah-Joël&segertno&596234
		\end{tabular}
	\end{center}
	\section*{Aufgabe 1}
		\subsection*{Direct Mapped Cache}
		\begin{longtable}{|cccccccc|}
			\hline
			0&1&2&3&4&5&6&7\\
			\hline
			&&2&&&&&\\
			\hline
			&&2&&&5&&\\
			\hline
			0&&2&&&5&&\\
			\hline
			0&&2&&&13&&\\
			\hline
			0&&[\textbf{2}]&&&13&&\\
			\hline
			0&&2&&&5&&\\
			\hline
			0&&10&&&5&&\\
			\hline
			8&&10&&&5&&\\
			\hline
			0&&10&&&5&&\\
			\hline
			0&&10&&4&5&&\\
			\hline
			0&&10&&4&[\textbf{5}]&&\\
			\hline
			0&&2&&4&5&&\\
			\hline 
		\end{longtable}
		In der Grafik sind \textbf{Hits} dick gedruckt in [ ] dargestellt.
		Bei gegebener Zugriffssequenz ergibt sich eine Trefferquote von \(\frac{2}{12} \approx 16.67\% \).
		
		\subsection*{Set-assoziativer Cache mit Set-Größe 4}
		Hierbei wird von der \textbf{FIFO}-Ersatzstrategie ausgegangen.\\
		\begin{longtable}{|cccc||cccc|}
			\hline
			0&1&2&3&0&1&2&3\\
			\hline
			2&&&&&&&\\
			\hline
			2&&&&5&&&\\
			\hline
			2&0&&&5&&&\\
			\hline
			2&0&&&5&13&&\\
			\hline
			[\textbf{2}]&0&&&5&13&&\\
			\hline
			2&0&&&[\textbf{5}]&13&&\\
			\hline
			2&0&10&&5&13&&\\
			\hline
			2&0&10&8&5&13&&\\
			\hline
			2&[\textbf{0}]&10&8&5&13&&\\
			\hline
			4&0&10&8&5&13&&\\
			\hline
			4&0&10&8&[\textbf{5}]&13&&\\
			\hline
			4&2&10&8&5&13&&\\
			\hline 
		\end{longtable}
		In der Grafik sind \textbf{Hits} dick gedruckt in [ ]   dargestellt.
		Bei gegebener Zugriffssequenz ergibt sich eine Trefferquote von \(\frac{4}{12} \approx 33.34\% \).
		
		\subsection*{Vollassoziativer Cache mit LRU-Zugriff}
		
		\begin{longtable}{|cccccccc|}
			\hline
			0&1&2&3&4&5&6&7\\
			\hline
			2&&&&&&&\\
			\hline
			2&5&&&&&&\\
			\hline
			2&5&0&&&&&\\
			\hline
			2&5&0&13&&&&\\
			\hline
			[\textbf{2}]&5&0&13&&&&\\
			\hline
			2&[\textbf{5}]&0&13&&&&\\
			\hline
			2&5&0&13&10&&&\\
			\hline
			2&5&0&13&10&8&&\\
			\hline
			2&5&[\textbf{0}]&13&10&8&&\\
			\hline
			2&5&0&13&10&8&4&\\
			\hline
			2&[\textbf{5}]&0&13&10&8&4&\\
			\hline
			[\textbf{2}]&5&0&13&10&8&4&\\
			\hline 
		\end{longtable}
		In der Grafik sind \textbf{Hits} dick gedruckt in [ ] dargestellt.
		Bei gegebener Zugriffssequenz ergibt sich eine Trefferquote von \(\frac{5}{12} \approx 41.67\% \).
	\section*{Aufgabe 2}
		Bei einem Writeback-Cache mit einer Cachelinegröße von 32 Byte, einer Größe von 1 MByte und einem byteadressierten Hauptspeicher der Größe 4 GByte, müssen:
		\subsection*{(a)}
		$\dotsc$ bei einem 4-fach-setassoziativen Cache 16 Bit für jeden Eintrag von Steuerinformationen bereitstehen.
		
		\subsection*{(b)}
		$\dotsc$ 29 Bit für Einträge der Steuerinformationen in einem voll assoziativen Cache bereitgestellt werden.
		
		\subsection*{(c)}
		$\dotsc$ 14 Bit bereitgestellt werden, wenn ein direct mapped Cache zum Einsatz kommt.
	\newpage
	\section*{Aufgabe 3}
		\subsection*{(a)}
		\begin{longtable}{|l|l|c|r|}
			\hline
			Adresse&Adresse binär&Seite&Offset\\\hline\hline
			  81&0000 0101 0001&0&81\\\hline
			1852&0111 0011 1100&3&316\\\hline
			 396&0001 1000 1100&0&396\\\hline
			2810&1010 1111 1010&5&372\\\hline
			2019&0111 1110 0011&3&483\\\hline
			 562&0010 0011 0010&1&25\\\hline
			3456&1101 1000 0000&6&192\\\hline
		\end{longtable}
		%9 bit offset
		\subsection*{(b)}
%		\begin{longtable}{|c|}
%			%Seitennummer&Pageframe\\\hline
%			\hline
%			000 0101 0001\\\hline
%			011 0011 1100\\\hline
%			101 1000 1100\\\hline
%			110 1111 1010\\\hline
%			001 1110 0011\\\hline
%			010 0011 0010\\\hline
%			101 1000 0000\\\hline
%		\end{longtable}
		\begin{longtable}{|c|c|}\hline
			Page frame&Valid Flag\\\hline
			\hline
			10&0\\\hline
			01&1\\\hline
			&0\\\hline
			00&1\\\hline
			&0\\\hline
			11&1\\\hline
			10&1\\\hline
			&0\\\hline
		\end{longtable}
		\subsection*{(c)}
		\begin{longtable}{|c|c|}
			%Seitennummer&Pageframe\\\hline
			\hline
			Log.Adresse&Phys. Adresse()\\\hline\hline
			000 0 0101 0001&00 0 0101 0001\\\hline
			011 1 0011 1100&01 1 0011 1100\\\hline
			000 1 1000 1100&10 1 1000 1100\\\hline
			101 0 1111 1010&11 0 1111 1010\\\hline
			011 1 1110 0011&00 1 1110 0011\\\hline
			001 0 0011 0010&01 0 0011 0010\\\hline
			110 1 1000 0000&10 1 1000 0000\\\hline
		\end{longtable}
\end{document}
